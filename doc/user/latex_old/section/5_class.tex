\section{Class Reference}

% In both user and developer manual
\subsection{Class Agent}
\textbf{Declared in}: src/agent/agent.h

\begin{apient}
void SetInitEntry(string funcd_name);
\end{apient}
\apidesc{

  Sets the entry payload function. The payload function should be linked to the
  agent library or should be in the agent library. By default, the entry payload
  function is ``default\_entry'', which simply prints out all executed
  functions' name.

}


\begin{apient}
void SetInitExit(string func_name);
\end{apient}
\apidesc{

  Sets the exit payload function. By default, there is not exit payload
  function.

}

\begin{apient}
void SetLibrariesToInstrument(const StringSet& libs);
\end{apient}
\apidesc{

  Sets the set of names of libraries loaded in the application process that will
  be instrumented. By default, we only instrument the executable binary code and
  don't instrument loaded shared libraries.

}

\begin{apient}
void SetFuncsNotToInstrument(const StringSet& funcs);
\end{apient}
\apidesc{

  Sets the set of names of functions in the application process that will NOT be
  instrumented. By default, we instrument all functions.

}


\begin{apient}
void EnableParseOnly(const bool yes_or_no);
\end{apient}
\apidesc{

  Enables or disables ParseOnly option. If yes\_or\_no is true, then after
  parsing the binary code, we don't do instrumentation; otherwise,
  instrumentation is conducted after parsing. By default, ParseOnly option is
  disabled.

}

\begin{apient}
void EnableIpc(const bool yes_or_no);
\end{apient}
\apidesc{

  Enables or disables inter-process instrumentation propagation. By default,
  inter-process instrumentation propagation is disabled.

}

\begin{apient}
void EnableHandleDlopen(const bool yes_or_no);
\end{apient}
\apidesc{

  Enables or disables propagating instrumentation to library loaded via
  dlopen. By default, it is disabled.

}

\begin{apient}
void EnableMultithread(const bool yes_or_no);
\end{apient}
\apidesc{

  Enables or disables propagating instrumentation across thread. By default, it
  is disabled.

}

\begin{apient}
void Go();
\end{apient}
\apidesc{

  Starts initial instrumentation.

}

\subsection{Class SpPoint}
\textbf{Declared in}: src/agent/patchapi/point.h

\begin{apient}
bool tailcall();
\end{apient}
\apidesc{

Indicates whether or not the function call at this point is a tail call.

}

\begin{apient}
SpBlock* GetBlock() const;
\end{apient}
\apidesc{

Returns the call block at this point.

}

\begin{apient}
SpObject* GetObject() const;
\end{apient}
\apidesc{
Returns the binary object at this point.
}


\subsection{Class SpObject}
\textbf{Declared in}: src/agent/patchapi/object.h

\begin{apient}
std::string name() const;
\end{apient}
\apidesc{
Returns the binary object's name.
}

This class inherits PatchAPI::PatchObject. Please refer to PatchAPI document for
the complete list of methods.

\subsection{Class SpFunction}
\textbf{Declared in}: src/agent/patchapi/cfg.h

\begin{apient}
SpObject* GetObject() const;
\end{apient}
\apidesc{
Returns the binary object containing this function.
}


\begin{apient}
std::string GetMangledName();
\end{apient}
\apidesc{
Returns the mangled name of this function.
}

\begin{apient}
std::string GetPrettyName();
\end{apient}
\apidesc{
Returns the demangled name of this function.
}

\begin{apient}
std::string name();
\end{apient}
\apidesc{
Returns the demangled name of this function, the same as calling GetPrettyName().
}

This class inherits PatchAPI::PatchFunction. Please refer to PatchAPI document
for the complete list of methods.

\subsection{Class SpBlock}
\textbf{Declared in}: src/agent/patchapi/cfg.h

\begin{apient}
SpObject* GetObject() const;
\end{apient}
\apidesc{
Returns the binary object containing this block.
}


\begin{apient}
in::Instruction::Ptr orig_call_insn() const;
\end{apient}
\apidesc{

Returns an InstructionAPI::Instruction instance of the call instruction in this
block.

}

This class inherits PatchAPI::PatchBlock. Please refer to PatchAPI document for
the complete list of methods.

\subsection{Class SpEdge}
\textbf{Declared in}: src/agent/patchapi/cfg.h

This class inherits PatchAPI::PatchEdge. Please refer to PatchAPI document for
the complete list of methods.

\subsection{Utility Functions}
\textbf{Declared in}: src/agent/payload.h

Utility functions are used when writing payload functions.

\begin{apient}
SpFunction* Callee(SpPoint* pt);
\end{apient}
\apidesc{

  Returns an instance of SpFunction for current function call.  The parameter
  \emph{pt} represents the instrumentation point for current function call.  If
  it fails (e.g., cannot find a function), it returns NULL.

}

\begin{apient}
bool IsInstrumentable(SpPoint* pt);
\end{apient}
\apidesc{
Indicates whether or not the point is instrumentable.
}

\begin{apient}
void Propel(SpPoint* pt);
\end{apient}
\apidesc{

Propagates instrumentation to callees of the function called at the specified
point {\em pt}.

}

\begin{apient}
bool IsIpcWrite(SpPoint* pt); 
bool IsIpcRead(SpPoint* pt); 
\end{apient}
\apidesc{

Indicates whether or not the function called at the specified point is a
inter-process communication write / read function.

}

\begin{apient}
struct ArgumentHandle;
void* PopArgument(SpPoint* pt,
                  ArgumentHandle* h,
                  size_t size);
\end{apient}
\apidesc{

Gets a pointer to an parameter of the function call at the specified point {\em
  pt}.
All parameters passed to the function call are in a stack associated with the
ArgumentHandle structure {\em h}. 
Leftmost parameter is at the top of the stack.
The {\em size} parameter specifies the size of the parameter that is about to be
popped.

}

\begin{apient}
long ReturnValue(SpPoint* pt);
\end{apient}
\apidesc{
Returns the return value of the function call at the specified point {\em pt}.
}



